\section{Introduction}
\label{section:introduction}

This paper describes a set of lectures on a collection of techniques and tools that can be used for engineering bidirectional transformations (BX). The motivation for these lectures is our view that transformations in general -- and BX in the specific -- are like other software systems: they are designed to be executed on a machine, are complicated (and in some cases even complex), and are difficult to build correctly. As such, like software, transformations should be engineered by following a rigorous process. The advantages of doing so are the same as for software, including:
\begin{itemize}
\item \textit{Repeatability:} by following a process, we potentially make it easier for others to repeat our work, or to reduce the amount of effort required to build a similar system in the future.
\item \textit{Review and Scale:} by decomposing a large engineering problem into stages, we potentially make it easier to audit and validate the results of each stage, and to solve larger problems than we would be able to if we treated the problem monolithically.
\item \textit{Automation:} by following a process we have greater opportunities to automate parts of it, e.g., generation of code or documents. 
\item \textit{Training:} by following and documenting a rigorous engineering process we may make it easier to train others.
\end{itemize}
BX are special kinds of transformations with, in our opinion, complicated execution semantics. As such, BX may especially benefit from following a repeatable, reviewable, scalable, automated process with training/guidance, for their development.

\subsection{BX as software}
The assumption that we are making in the preceding is that BX are software systems. A software system is an executable artefact: given a specification of software (e.g., in a programming language or suitable modelling language), its expected outputs can be produced by executing the specification on a suitable machine (e.g., a server, a virtual machine, a simulator). A BX is an executable artefact: assuming that the BX is expressed in a suitable programming language or modelling language (and we review some of the key state of the art in Section~\ref{section:state-of-the-art}) then its expected outputs can be produced by executing the BX on a suitable machine. 

Like software, BX must satisfy functional and non-functional requirements, can (and probably should) be designed, and can exhibit unacceptable behaviour -- that is, BX can contain bugs or faults, which may lead to failures. As we will see, depending on the technologies used to represent and specify BX, different types of failures may arise (e.g., inconsistencies) and different techniques may be used to verify the BX to help ensure that faults are caught during engineering. As we become increasingly ambitious in our attempts to solve complex problems using BX, our need for rigorous engineering techniques for BX construction will only increase.

\subsection{Scope}
There are numerous techniques and approaches that can be used to build and engineer BX; in Section~\ref{section:state-of-the-art} we will consider some of these. However, the focus of this paper will be on Model-Driven Engineering (MDE) techniques, because of our experience with them. Many of the techniques that we present in later sections can be used both with and without MDE tools, and if there are particular aspects that depend specifically on MDE, we will point these out where such a dependence isn't clear.

\subsection{Background}
Before we commence with the technical content of this paper, we provide some basic definitions and terminology, in order that the paper remain reasonably self-contained.

As mentioned, we are focusing on Model-Driven Engineering techniques for engineering BX. The key concepts of MDE are as follows.

\begin{itemize}
\item MDE involves the semi-automated construction and manipulation of \textit{models}, which are structured, machine-implemented specifications of phenomena of interest. Models are meant to be processable by automated tools, and capture static and dynamic characteristics of systems. 

\item Models in MDE are \textit{structured}; this structure can be defined in a number of ways, including via \textit{metamodels}, which are specifications of abstract syntax (you can think of a metamodel as a the definition of the abstract syntax of a language). 

A model is said to \textit{conform} to a metamodel. Related approaches to defining the structure of models include schemas (e.g., XML), type rules and constraints. Many of these approaches define structure using graphs or graph-like concepts. As such, models themselves are typically graphs. This is a key distinction between MDE (and so-called \textit{modelware} approaches to engineering), and grammar-based (or \textit{grammarware}) approaches.

\item Models are typically specified alongside a set of \textit{constraints} that capture well-formedness rules that cannot normally be specified with a metamodel. For example, a metamodel might be used to express that a model may include containers, and that containers may be nested (e.g., packages in Java or UML). But a metamodel -- which captures abstract syntax -- will not normally express that containers have unique names. This can be expressed by a separate constraint, which is normally packaged up with the metamodel or models. If a model conforms to a metamodel, it must also normally be checked against any constraints, in order to establish that it is well formed.

\item Standard technologies exist for capturing models, metamodels and constraints in the MDE world. The de facto standard technology used for metamodelling is Ecore (a part of the Eclipse Modelling Framework (EMF)). For constraints, engineers typically use the Object Constraint Language (OCL), which also has an official Eclipse implementation. There are other languages and technologies available as well (e.g., typed graphs, MetaDepth, XMI) for metamodelling and for expressing constraints.

\item Models by themselves typically encapsulate business value, but are also meant to be processed by automated tools. These tools implement a variety of operations applicable to models, including the aforementioned transformations, but also comparisons, merging, migration, matching and others.
\end{itemize}

Transformations are a key operation in MDE, and have been the subject of widespread study (e.g., see the proceedings of the long-running conference on model transformation, such as \cite{icmt2016} ). Numerous classifications and surveys have been published on transformations in general, and BX in the specific. Four common categories of transformations in MDE are:
\begin{itemize}
\item \textit{Unidirectional transformations}, from a source model to a target model. Such transformations are usually implemented in terms of metamodels, and are typically used when the source and target metamodel are linguistically similar, e.g., between different dialects of UML, or from an object-oriented model to a relational database model. Unidirectional transformations typically are written in one of three styles: purely declarative, operational, and hybrid (i.e., a mixture of operational and declaration parts). In our experience, many complicated transformations are very difficult to express in a purely declarative style, and as such hybrid unidirectional transformation languages (such as ATL \cite{JouaultABK08} and ETL \cite{Kolovos2008}) tend to see the most use in practice.

\item \textit{Update-in-place transformations}, which specify modifications made to one and only one model. Update-in-place transformations can be specified using languages suitable for unidirectional transformations, or specialist languages such as EWL \cite{KolovosPPR07}.

\item \textit{Model-to-text} (sometimes called model-to-grammar) transformations, where the source/input to the transformation is a model, but the output is no longer a model, e.g., free-form text or text conforming to a grammar. Model-to-text transformations are used in order to step outside of the modelware technical space and move to the grammarware technical space. An example scenario for use of model-to-text transformation is code generation.

\item Bidirectional transformations, which is the subject of the next section.
\end{itemize}
Transformations (and other operations on models) have side-effects. This includes purely declarative transformations. The side-effect in question is the production of \textit{traceability} information, i.e., so-called \textit{trace-links}, which relate source and target model elements. Trace-links can be generated automatically by transformation tools (such as Epsilon or ATL) and they can be stored for later audit and analysis. Trace-links are important in the context of transformations and BX as they provide (a) the basis for verification and validation of transformations; and (b) the connection to the theory behind BX, specifically delta lenses (in particular, delta lenses are a sound theory for trace models, encoded in an algebraic form \cite{DiskinXC10}).

\subsection{Structure of paper}
The paper starts with a brief review of the state-of-the-art in engineering BX with MDE, focusing firstly on BX scenarios of use in MDE, followed by an overview of MDE languages, tools and techniques for supporting BX. The remainder of the paper considers different aspects of a BX engineering lifecycle, starting with an overview of techniques for requirements engineering for BX, focusing on requirements specification and requirements analysis. We then move to an overview of techniques for architecture and design of BX, including a small selection of relevant design patterns. Finally, we briefly consider one approach for verification of BX, which applies to a specific approach to BX implementation and design. The paper concludes with a discussion on future challenges and perspectives on engineering of BX.

