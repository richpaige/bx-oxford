\section{Conclusions and Perspectives}
\label{section:conclusions}

Bidirectional transformations must be engineered, as must unidirectional transformations and other programs that manipulate models in MDE. The state-of-the-art in engineering BX is piecemeal at the moment: there are some specific techniques for supporting different engineering phases -- such as requirements engineering or design -- but very coarse understanding of efficient and effective engineering lifecycles, and alternative process models. This paper attempts to capture some of the current thinking on engineering BX. It summarises some of the state-of-the-art in BX design and implementation, presents some approaches for requirements specification and analysis, and suggests some ideas for capturing the architecture of complicated BX, and the detailed design of BX in general. It also presents some ideas on an approach for verification of BX; this approach is pragmatic, in the sense that it is meant to be used within an engineering process and it acknowledges tradeoffs between completeness and soundness.

MDE for BX possesses some sound theory -- such as delta lenses -- and some pragmatic, if incomplete, tools (such as Eclipse QVT-Relations) but these are still siloed: the theory needs to inform the enhancement of tools, and the tools need to be used to test the corners of the theory. A good example of research that attempts to link BX theory and practice is that combining triple graph grammars and delta lenses (e.g., \cite{0001EOCDXGE15}), but more needs to be done. What is really needed is tools that \textit{evidently} implement the theory in a systematic and audited way.

A key challenge in connecting theory with practical tools is the limitations in our theories of metamodelling. It is questionable whether we have a sound and complete understanding of a type theory for MDE and metamodelling, but this would underpin any attempts to link a theory of BX with the pragmatic tools supporting BX.

We mentioned tools for BX throughout this paper. The standardised tool in the MDE community is QVT-Relations; the Eclipse implementation is still under development. QVT-Relations has been criticised for being very complex, with substantial semantic ambiguity. The development of its Eclipse implementation is revealing some of these ambiguities, but this will only be convincing if supported by a sound theory, e.g., delta lenses. However, the gap between delta lens theory and QVT-Relations is substantial: changing QVT-Relations to conform with delta lenses may be difficult if not impossible; building a new BX that supports delta lens theory is possible, but it would not be QVT-Relations. It is difficult to see how connections between strong theory and MDE standards will play out. 

It also remains to be seen whether we can develop a rich, compelling set of industrial scenarios for BX. In our substantial industrial experience of transformations and MDE, we have had only one precise requirement for a BX (across over 20 industrial projects and 13 years of experience), and that was for the results of various forms of analysis (e.g., failure analysis, performance analysis) to be reflected on source models after calculation. It is unclear if such scenarios benefit from the heavyweight machinery of BX. But it should also be noted that requirements for BX sometimes emerge as development proceeds and having ways in which transformations can be \textit{extended} to become bidirectional may be useful.

In Section~\ref{section:verification} we described an approach to BX that involved specification of inter-model consistency constraints between two models, and the definition of two separate but synchronised update-in-place transformations on the two models. When the constraints were violated and the models became inconsistent, the transformations would be triggered to re-establish consistency. This approach -- two simple yet unidirectional transformations instead of a single bidirectional transformation -- needs to be clearly related to the BX solution space: when is it more effective to use versus building a full BX?

Finally, we observe that many transformations developed in practice are operational (e.g., those written in EOL or subsets of ATL). As well, there are many model-to-text (or model-to-grammar) transformations that support code generation scenarios. How do these fit in to the BX space? Are they simply too hard to consider? Are there scenarios or types of transformations that simple \textit{should not} (rather than \textit{cannot}) be bidirectionalised? As a challenge, consider the EuGENia tool\footnote{\url{https://eclipse.org/epsilon/doc/eugenia/}} which is a unidirectional model transformation written in EOL, which automatically generates three models needed by GMF to construct a graphical editor. These are generated by a transformation that takes as input a single annotated Ecore model. The transformation is defined entirely operationally, as we found that it would be too complex to implement using declarative rules (it is not a mapping transformation). Could EuGENia be turned into a bidirectional transformation? Our intuition is no (and, more pragmatically, we cannot see any reason why one would want to do so), but it would be interesting to explore what, fundamentally, makes an operational or hybrid transformation difficult to bidirectionalise.

